\begin{frame}{Conclusões}
    \begin{itemize}
        \item Neste trabalho, foi desenvolvido um software para automatizar a implementação e simulação de modelos computacionais baseados em EDOs.
        \item A construção das equações do modelo matemático é auxiliada pela representação visual que foi criada permitindo que o usuário acompanhe a construção de todas as expressões e como elas estão sendo combinadas para formar o sistema de EDOs.
        \item Através da GUI, é possível ver as entradas e operações de cada expressão, os sinais de cada entrada, quais expressões fazem parte de uma determinada EDO, entre outras coisas. 
    \end{itemize}
\end{frame}

\begin{frame}{Conclusões}
    \begin{itemize}
        \item Como limitações do trabalho, destaca-se: 
        \begin{itemize}
            \item A representação visual apresenta uma limitação na qual tornar-se mais difícil entender um modelo complexo com muitos nós e ligações. 
            \item Não foi realizada uma avaliação de usabilidade do software. 
        \end{itemize}
    \end{itemize}
\end{frame}

\begin{frame}{Trabalhos futuros}
    \begin{itemize}
        \item Como trabalhos futuros, destaca-se: 
        \begin{itemize}
            \item Geração de código e simulação de modelos estocásticos;
            \note[item]{Modelos estocásticos: utilizando o algoritmo de Gillespie, o mesmo editor de nós e um novo template para a simulação}
            \item Ajustes de parâmetros; 
            \note[item]{Ajustes de parâmetros: fornecendo recursos para o carregamento de dados experimentais, escolha dos parâmetros a serem ajustados e plotagens comparativas}
            \item Análise de sensibilidade de parâmetros;
            \item Geração de código e simulação de Equações Diferenciais Parciais (EDPs);
            \item Desenvolvimento de uma versão Web do software.
            \note[item]{Web: Rust e as tecnologias usadas na interface gráfica possuem suporte nativo à web (via WebAssebmly). Existem distribuições de Python suportadas no navegador, mas uma solução poderia envolver a execução das simulações do lado do servidor ao invés do cliente.}
        \end{itemize}
    \end{itemize}
\end{frame}