\begin{frame}{Motivação}
    \begin{itemize}
        \item \textbf{Objetivo}: desenvolver um software para facilitar a construção e simulação de modelos matemáticos e computacionais;
        \begin{itemize}
            \item Romper a barreira envolvida: não-programadores também devem conseguir usufruir do software;
        \end{itemize}
        \item \textbf{Como}: utilizando uma representação fácil de entender, flexível para manutenção e eficiente em computação;
        \begin{itemize}
            \item E mais: ofertar ao usuário a opção de exportar o modelo como código. Assim, usuários mais avançados (ou da área) poderão usufruir de modificar o modelo a seus gostos, enquanto usuários iniciantes poderão aprender a construir seus próprios;
        \end{itemize}
    \end{itemize}
\end{frame}


\subsection{Referencial Teórico}

\begin{chapter}[beamerthemesrc/assets/background_negative]{}{Referencial Teórico}
\end{chapter}

\begin{sidepic}{beamerthemesrc/images/gol-cut}{Autômatos Celulares}
   \begin{itemize}
       \item Consiste em uma grade de células, cada uma em um estado de um conjunto finito;
        \item As células transitam entre estados utilizando uma função que recebe os estados de seus vizinhos como entrada;
        \item O objetivo é estudar a evolução do sistema com o tempo;
        \item Com poucas regras é possível criar comportamentos complexos;
    \end{itemize}
\end{sidepic}

\begin{frame}{Equações Diferenciais Ordinárias (EDOs)}
    \begin{itemize}
        \item Usadas para estudar o comportamento populacional ao longo do tempo;
        \item Diversas aplicações em várias áreas do conhecimento; 
        \item Cada equação descreve a concentração de uma população diferente;
    \end{itemize}

    \begin{columns}
        \begin{column}{.4\textwidth}
            \begin{equation}
                \begin{array}{lr}
                    \frac{dN_1}{dt} = W_{11}.N_1 + W_{21}.N_2
                    \\
                    \\
                    \frac{dN_2}{dt} = W_{22}.N_2 + W_{12}.N_1
                \end{array}
            \end{equation}
        \end{column}

        \begin{column}{.6\textwidth}
            \begin{figure}
                \centering
                \includegraphics[height=.7\textheight]{beamerthemesrc/images/ode}
            \end{figure}
        \end{column}
    \end{columns}
\end{frame}

\begin{frame}{EDO — Modelo Predador-Presa}
    Um modelo clássico da literatura é o modelo Predador-Presa. Este modelo descreve o comportamento de duas populações, $H$ e $P$, que possuem uma uma relação de predação entre si. 

    \begin{columns}
        \begin{column}{.3\textwidth}
            \begin{equation}\label{eq:predadorpresa}
                \begin{array}{lr}
                    \frac{dH}{dt} = r.H - a.H.P
                    \\
                    \\
                    \frac{dP}{dt} = b.H.P - m.P
                \end{array}
            \end{equation}
        \end{column}
        \begin{column}{.6\textwidth}
            Na equação, temos que
            \[
            \begin{array}{lr}
                H & \text{Presa}\\
                P & \text{Predador}\\
                r & \text{Taxa de reprodução da presa}\\
                m & \text{Taxa de mortalidade dos predadores}\\
                a & \text{Taxa de predação}\\
                b & \text{Taxa de reprodução dos predadores}\\
                \end{array}.
            \]
        \end{column}
    \end{columns}

\end{frame}
