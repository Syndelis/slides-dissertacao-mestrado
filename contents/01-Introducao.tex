\begin{frame}{Introdução}
    \begin{itemize}
        \item O desenvolvimento de modelos computacionais requer um conjunto de etapas: estudo do problema, formulação de hipóteses, construção, implementação e simulação do modelo. 
        \item Uma das etapas mais desafiadoras é a implementação.
        \begin{itemize}
            \item Requer o conhecimento de programação, estrutura de dados, bibliotecas, etc; 
            \item Um erro na implementação pode comprometer todo o trabalho. 
        \end{itemize}
        \item Questão científica: 
        \begin{itemize}
            \item É possível que ferramentas de software automatizem etapas do processo de modelagem computacional?
        \end{itemize}
    \end{itemize}
\end{frame}

\begin{frame}{Introdução}
    \begin{itemize}
        \item Para responder a questão anterior, foi desenvolvido um software para auxiliar a implementação e simulação de modelos de Equações Diferenciais Ordinárias (EDOs) com o objetivo de automatizar certas etapas do processo de modelagem. 
        \item A partir de uma representação visual de um modelo, o software é capaz de gerar o código que implementa o modelo, simulá-lo e até exportar gráficos com os resultados.
        \item O software foi desenvolvido com os seguintes objetivos:
        \begin{itemize}
            \item Aplicação em pesquisas;
            \item Auxílio no ensino-aprendizagem de modelagem computacional;
        \end{itemize}
    \end{itemize}
\end{frame}
