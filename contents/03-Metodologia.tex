\begin{frame}{Fluxograma de uso simplificado}
    \begin{figure}
        \centering
        \includegraphics[width=\textwidth, height=\textheight, keepaspectratio=true]{beamerthemesrc/images/ode-designer-fluxograma.png}
        \caption{Fluxograma da experiência do usuário.}
    \end{figure}
\end{frame}

\begin{frame}{Funcionalidades do Software}
    O software deve entregar as seguintes funcionalidades:
    \begin{itemize}
        \item Criação de modelos pela interface gráfica;
        \item Simulação do modelo e exibição dos resultados na interface;
        \item Exportação de PDF/Imagens com os resultados das simulações;
        \item Exportação de código equivalente ao modelo implementado;
    \end{itemize}
\end{frame}

\subsection{Interface Gráfica}

\begin{sidepic}{beamerthemesrc/images/ode-designer-gui-exemplo}{Representação do Modelo}
    \begin{itemize}
        \item O software necessita de uma interface simples de ser usada;
            \begin{itemize}
                \item Mas também deve naturalmente relembrar uma EDO;
            \end{itemize}
        \item Após diversas iterações, chegamos numa interface baseada em programação visual;
            \begin{itemize}
                \item Estas interfaces existem desde 1963, mas tiveram uma renascença com o avanço dos computadores e a necessidade de softwares de edição de imagem/áudio/vídeo;
            \end{itemize}
    \end{itemize}
\end{sidepic}

\begin{frame}{Fluxo de passagem de informações na interface}
    \begin{figure}
        \centering
        \includegraphics[width=\textwidth, height=\textheight, keepaspectratio=true]{beamerthemesrc/images/fluxo-dados-gui.png}
        \caption{Relações entre nós e pinos.}
    \end{figure}
\end{frame}

\begin{frame}{GRaIL — 1968}
    \href{https://www.youtube.com/watch?v=QQhVQ1UG6aM}{
        \includegraphics[width=\textwidth, height=\textheight, keepaspectratio=true]{beamerthemesrc/images/video-thumb.png}
    }
\end{frame}

\subsection{Representação Intermediária}

\begin{chapter}[beamerthemesrc/assets/background_negative]{}{Representação Intermediária}
\end{chapter}

\begin{frame}{Representação Intermediária}
    \begin{itemize}
        \item Com o objetivo de realizar tantas transformações, torna-se necessário a utilização de uma Representação Intermediária (RI);

        \item Inspirados nas arquiteturas de compiladores modernos (GCC, baseados em LLVM), separamos a estrutura em back-end e front-end;

        \item Essa abordagem garante o desacoplamento entre estrutura e produtos finais;
    \end{itemize}

    \begin{figure}
        \centering
        \includegraphics[height=\textheight, width=\textwidth, keepaspectratio=true]{beamerthemesrc/images/llvm-ir.png}
        \caption{Exemplo de RI: LLVM-IR.}
    \end{figure}
\end{frame}

\begin{frame}[fragile]{RI — Serde: Conversões automatizadas para JSON}
    \begin{columns}
        \begin{column}{.55\textwidth}
            \begin{block}{Rust}
                \begin{lstlisting}[language=Rust]
#[derive(Serialize, Deserialize)]
struct Person {
    name: String,
    age: u8,
    phones: Vec<String>,
    address: Address,
}
#[derive(Serialize, Deserialize)]
struct Address {
    street: String,
    city: String,
}
                \end{lstlisting}
            \end{block}
        \end{column}

        \begin{column}{.45\textwidth}
            \begin{block}{JSON}
                \begin{lstlisting}[language=json, tabsize=2]
{
  "name": "John Doe",
  "age": 43,
  "address": {
    "street": "1st St.",
    "city": "London"
  },
  "phones": [
    "+44 1234567",
    "+44 2345678"
  ]
}
                \end{lstlisting}
            \end{block}
        \end{column}
    \end{columns}
\end{frame}

\begin{frame}[fragile]{RI — \textit{Templates}}

    \begin{itemize}
        \item \texttt{Structs} \textit{serializáveis} podem ser usadas diretamente em \textit{templates};
        \item Suponha a variável \texttt{people: Vec<Person>}:
    \end{itemize}

    \begin{block}{minijinja}
        \begin{lstlisting}[language=jinja2]

    {{ person.name }}, {{ person.age }} anos.
    Contato: 
      {{ phone_nb }}
      , 
    

        \end{lstlisting}
    \end{block}
\end{frame}

\begin{frame}{Fluxo de conversões}
    \begin{figure}
        \centering
        \includegraphics[width=\textwidth, height=\textheight, keepaspectratio=true]{beamerthemesrc/images/fluxo-conversoes.png}
        \caption{Fluxograma das transformações realizadas pelo software.}
    \end{figure}
\end{frame}
